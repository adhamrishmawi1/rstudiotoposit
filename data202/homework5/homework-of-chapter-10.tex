% Options for packages loaded elsewhere
\PassOptionsToPackage{unicode}{hyperref}
\PassOptionsToPackage{hyphens}{url}
%
\documentclass[
]{article}
\title{DATA 202 Homework 4}
\author{Adham Rishmawi}
\date{2022-10-29}

\usepackage{amsmath,amssymb}
\usepackage{lmodern}
\usepackage{iftex}
\ifPDFTeX
  \usepackage[T1]{fontenc}
  \usepackage[utf8]{inputenc}
  \usepackage{textcomp} % provide euro and other symbols
\else % if luatex or xetex
  \usepackage{unicode-math}
  \defaultfontfeatures{Scale=MatchLowercase}
  \defaultfontfeatures[\rmfamily]{Ligatures=TeX,Scale=1}
\fi
% Use upquote if available, for straight quotes in verbatim environments
\IfFileExists{upquote.sty}{\usepackage{upquote}}{}
\IfFileExists{microtype.sty}{% use microtype if available
  \usepackage[]{microtype}
  \UseMicrotypeSet[protrusion]{basicmath} % disable protrusion for tt fonts
}{}
\makeatletter
\@ifundefined{KOMAClassName}{% if non-KOMA class
  \IfFileExists{parskip.sty}{%
    \usepackage{parskip}
  }{% else
    \setlength{\parindent}{0pt}
    \setlength{\parskip}{6pt plus 2pt minus 1pt}}
}{% if KOMA class
  \KOMAoptions{parskip=half}}
\makeatother
\usepackage{xcolor}
\IfFileExists{xurl.sty}{\usepackage{xurl}}{} % add URL line breaks if available
\IfFileExists{bookmark.sty}{\usepackage{bookmark}}{\usepackage{hyperref}}
\hypersetup{
  pdftitle={DATA 202 Homework 4},
  pdfauthor={Adham Rishmawi},
  hidelinks,
  pdfcreator={LaTeX via pandoc}}
\urlstyle{same} % disable monospaced font for URLs
\usepackage[margin=1in]{geometry}
\usepackage{color}
\usepackage{fancyvrb}
\newcommand{\VerbBar}{|}
\newcommand{\VERB}{\Verb[commandchars=\\\{\}]}
\DefineVerbatimEnvironment{Highlighting}{Verbatim}{commandchars=\\\{\}}
% Add ',fontsize=\small' for more characters per line
\usepackage{framed}
\definecolor{shadecolor}{RGB}{248,248,248}
\newenvironment{Shaded}{\begin{snugshade}}{\end{snugshade}}
\newcommand{\AlertTok}[1]{\textcolor[rgb]{0.94,0.16,0.16}{#1}}
\newcommand{\AnnotationTok}[1]{\textcolor[rgb]{0.56,0.35,0.01}{\textbf{\textit{#1}}}}
\newcommand{\AttributeTok}[1]{\textcolor[rgb]{0.77,0.63,0.00}{#1}}
\newcommand{\BaseNTok}[1]{\textcolor[rgb]{0.00,0.00,0.81}{#1}}
\newcommand{\BuiltInTok}[1]{#1}
\newcommand{\CharTok}[1]{\textcolor[rgb]{0.31,0.60,0.02}{#1}}
\newcommand{\CommentTok}[1]{\textcolor[rgb]{0.56,0.35,0.01}{\textit{#1}}}
\newcommand{\CommentVarTok}[1]{\textcolor[rgb]{0.56,0.35,0.01}{\textbf{\textit{#1}}}}
\newcommand{\ConstantTok}[1]{\textcolor[rgb]{0.00,0.00,0.00}{#1}}
\newcommand{\ControlFlowTok}[1]{\textcolor[rgb]{0.13,0.29,0.53}{\textbf{#1}}}
\newcommand{\DataTypeTok}[1]{\textcolor[rgb]{0.13,0.29,0.53}{#1}}
\newcommand{\DecValTok}[1]{\textcolor[rgb]{0.00,0.00,0.81}{#1}}
\newcommand{\DocumentationTok}[1]{\textcolor[rgb]{0.56,0.35,0.01}{\textbf{\textit{#1}}}}
\newcommand{\ErrorTok}[1]{\textcolor[rgb]{0.64,0.00,0.00}{\textbf{#1}}}
\newcommand{\ExtensionTok}[1]{#1}
\newcommand{\FloatTok}[1]{\textcolor[rgb]{0.00,0.00,0.81}{#1}}
\newcommand{\FunctionTok}[1]{\textcolor[rgb]{0.00,0.00,0.00}{#1}}
\newcommand{\ImportTok}[1]{#1}
\newcommand{\InformationTok}[1]{\textcolor[rgb]{0.56,0.35,0.01}{\textbf{\textit{#1}}}}
\newcommand{\KeywordTok}[1]{\textcolor[rgb]{0.13,0.29,0.53}{\textbf{#1}}}
\newcommand{\NormalTok}[1]{#1}
\newcommand{\OperatorTok}[1]{\textcolor[rgb]{0.81,0.36,0.00}{\textbf{#1}}}
\newcommand{\OtherTok}[1]{\textcolor[rgb]{0.56,0.35,0.01}{#1}}
\newcommand{\PreprocessorTok}[1]{\textcolor[rgb]{0.56,0.35,0.01}{\textit{#1}}}
\newcommand{\RegionMarkerTok}[1]{#1}
\newcommand{\SpecialCharTok}[1]{\textcolor[rgb]{0.00,0.00,0.00}{#1}}
\newcommand{\SpecialStringTok}[1]{\textcolor[rgb]{0.31,0.60,0.02}{#1}}
\newcommand{\StringTok}[1]{\textcolor[rgb]{0.31,0.60,0.02}{#1}}
\newcommand{\VariableTok}[1]{\textcolor[rgb]{0.00,0.00,0.00}{#1}}
\newcommand{\VerbatimStringTok}[1]{\textcolor[rgb]{0.31,0.60,0.02}{#1}}
\newcommand{\WarningTok}[1]{\textcolor[rgb]{0.56,0.35,0.01}{\textbf{\textit{#1}}}}
\usepackage{graphicx}
\makeatletter
\def\maxwidth{\ifdim\Gin@nat@width>\linewidth\linewidth\else\Gin@nat@width\fi}
\def\maxheight{\ifdim\Gin@nat@height>\textheight\textheight\else\Gin@nat@height\fi}
\makeatother
% Scale images if necessary, so that they will not overflow the page
% margins by default, and it is still possible to overwrite the defaults
% using explicit options in \includegraphics[width, height, ...]{}
\setkeys{Gin}{width=\maxwidth,height=\maxheight,keepaspectratio}
% Set default figure placement to htbp
\makeatletter
\def\fps@figure{htbp}
\makeatother
\setlength{\emergencystretch}{3em} % prevent overfull lines
\providecommand{\tightlist}{%
  \setlength{\itemsep}{0pt}\setlength{\parskip}{0pt}}
\setcounter{secnumdepth}{-\maxdimen} % remove section numbering
\ifLuaTeX
  \usepackage{selnolig}  % disable illegal ligatures
\fi

\begin{document}
\maketitle

\#\#\#Excercise 1

\begin{enumerate}
\def\labelenumi{\alph{enumi}.}
\tightlist
\item
  because 27.2 is the average of all ages when gestation occurs so 27 is
  the mean of all those numbers not when the average age when bgestation
  occurs. b.Gestation is a period of time when the baby is developing so
  the statement insinuates that 27.2 is when birth occurs which is false
  assuming it is discussing average age when birth occurs.
\end{enumerate}

\#\#\#Excercise 2

\begin{Shaded}
\begin{Highlighting}[]
\NormalTok{gestation\_no\_missing }\OtherTok{\textless{}{-}}\NormalTok{ mosaicData}\SpecialCharTok{::}\NormalTok{Gestation }\SpecialCharTok{\%\textgreater{}\%} \FunctionTok{filter}\NormalTok{(}\SpecialCharTok{!}\FunctionTok{is.na}\NormalTok{(age))}
\FunctionTok{mean}\NormalTok{(}\SpecialCharTok{\textasciitilde{}}\NormalTok{ age, }\AttributeTok{data =}\NormalTok{ gestation\_no\_missing)}
\end{Highlighting}
\end{Shaded}

\begin{verbatim}
[1] 27.25527
\end{verbatim}

\begin{Shaded}
\begin{Highlighting}[]
\FunctionTok{set.rseed}\NormalTok{(}\DecValTok{123}\NormalTok{)}
\NormalTok{bootstrap }\OtherTok{\textless{}{-}}
  \FunctionTok{do}\NormalTok{(}\DecValTok{1000}\NormalTok{) }\SpecialCharTok{*} \FunctionTok{df\_stats}\NormalTok{(}\SpecialCharTok{\textasciitilde{}}\NormalTok{ age, }\AttributeTok{data =} \FunctionTok{resample}\NormalTok{(gestation\_no\_missing),}
\NormalTok{                      mean, }\AttributeTok{long\_names =} \ConstantTok{TRUE}\NormalTok{)}
\NormalTok{ci\_stats }\OtherTok{\textless{}{-}} \FunctionTok{cdata}\NormalTok{(}\SpecialCharTok{\textasciitilde{}}\NormalTok{ mean\_age, }\AttributeTok{data =}\NormalTok{ bootstrap, }\AttributeTok{p =} \FloatTok{0.95}\NormalTok{)}
\NormalTok{ci\_stats}
\end{Highlighting}
\end{Shaded}

\begin{verbatim}
        lower    upper central.p
2.5% 26.94652 27.57628      0.95
\end{verbatim}

Because for the 95\% samples from the population, the coinfidence
interval we compute will contain the true mean. Bootstrap resampling is
when we pretend that our sample is the actually population and
resampling from that fake population. The mean is a composition of the
average of histogram values so it makes sense that the intervals falls
in the middle to balance out the numbers that skew right and left

\hypertarget{excercise-3}{%
\subsubsection{Excercise 3}\label{excercise-3}}

\begin{Shaded}
\begin{Highlighting}[]
\NormalTok{model }\OtherTok{\textless{}{-}} \FunctionTok{gf\_point}\NormalTok{(wt }\SpecialCharTok{\textasciitilde{}}\NormalTok{ gestation, }\AttributeTok{data =}\NormalTok{ gestation\_no\_missing)}
\end{Highlighting}
\end{Shaded}

\begin{Shaded}
\begin{Highlighting}[]
\NormalTok{slr }\OtherTok{\textless{}{-}} \FunctionTok{lm}\NormalTok{(wt }\SpecialCharTok{\textasciitilde{}}\NormalTok{ gestation, }
          \AttributeTok{data =}\NormalTok{ gestation\_no\_missing)}

\FunctionTok{gf\_point}\NormalTok{(wt }\SpecialCharTok{\textasciitilde{}}\NormalTok{ gestation, }
         \AttributeTok{data =}\NormalTok{ gestation\_no\_missing) }\SpecialCharTok{|}\ErrorTok{\textgreater{}} 
  \FunctionTok{gf\_lm}\NormalTok{()}
\end{Highlighting}
\end{Shaded}

\begin{verbatim}
Warning: Removed 13 rows containing non-finite values (stat_lm).
\end{verbatim}

\begin{verbatim}
Warning: Removed 13 rows containing missing values (geom_point).
\end{verbatim}

\includegraphics{homework-of-chapter-10_files/figure-latex/B.-1.pdf}

\begin{Shaded}
\begin{Highlighting}[]
\FunctionTok{summary}\NormalTok{(slr)}
\end{Highlighting}
\end{Shaded}

\begin{verbatim}
Call:
lm(formula = wt ~ gestation, data = gestation_no_missing)

Residuals:
    Min      1Q  Median      3Q     Max 
-49.395 -11.145   0.105  10.140  57.356 

Coefficients:
             Estimate Std. Error t value Pr(>|t|)    
(Intercept) -10.07137    8.33033  -1.209    0.227    
gestation     0.46429    0.02977  15.595   <2e-16 ***
---
Signif. codes:  0 '***' 0.001 '**' 0.01 '*' 0.05 '.' 0.1 ' ' 1

Residual standard error: 16.68 on 1219 degrees of freedom
  (13 observations deleted due to missingness)
Multiple R-squared:  0.1663,    Adjusted R-squared:  0.1656 
F-statistic: 243.2 on 1 and 1219 DF,  p-value: < 2.2e-16
\end{verbatim}

The goal of a confidence interval method is to have the coverage rate to
be equal to the confidence level. So this means that the confidence
interval method is to be correct 95\% of the time. Gestation has low
coefficient of 0.405 and a high coefficient of 0.5226169 this mean the
mean of gestation occurs between those intervals at confidence level of
95\%.

\begin{Shaded}
\begin{Highlighting}[]
\NormalTok{alr }\OtherTok{\textless{}{-}} \FunctionTok{lm}\NormalTok{(wt }\SpecialCharTok{\textasciitilde{}}\NormalTok{ age, }
          \AttributeTok{data =}\NormalTok{ gestation\_no\_missing)}

\FunctionTok{gf\_point}\NormalTok{(wt }\SpecialCharTok{\textasciitilde{}}\NormalTok{ age, }
         \AttributeTok{data =}\NormalTok{ gestation\_no\_missing) }\SpecialCharTok{|}\ErrorTok{\textgreater{}} 
  \FunctionTok{gf\_lm}\NormalTok{()}
\end{Highlighting}
\end{Shaded}

\includegraphics{homework-of-chapter-10_files/figure-latex/unnamed-chunk-2-1.pdf}

\begin{Shaded}
\begin{Highlighting}[]
\FunctionTok{coef}\NormalTok{(alr)}
\end{Highlighting}
\end{Shaded}

\begin{verbatim}
(Intercept)         age 
116.6834606   0.1062233 
\end{verbatim}

This tells us the age of the mother did not influence the average weight
of child and it stayed fairly consistent!

\end{document}
